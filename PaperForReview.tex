% CVPR 2022 Paper Template
% based on the CVPR template provided by Ming-Ming Cheng (https://github.com/MCG-NKU/CVPR_Template)
% modified and extended by Stefan Roth (stefan.roth@NOSPAMtu-darmstadt.de)

\documentclass[10pt,twocolumn,letterpaper]{article}

%%%%%%%%% PAPER TYPE  - PLEASE UPDATE FOR FINAL VERSION
\usepackage[review]{cvpr}      % To produce the REVIEW version
%\usepackage{cvpr}              % To produce the CAMERA-READY version
%\usepackage[pagenumbers]{cvpr} % To force page numbers, e.g. for an arXiv version

% Include other packages here, before hyperref.
\usepackage{graphicx}
\usepackage{amsmath}
\usepackage{amssymb}
\usepackage{booktabs}


% It is strongly recommended to use hyperref, especially for the review version.
% hyperref with option pagebackref eases the reviewers' job.
% Please disable hyperref *only* if you encounter grave issues, e.g. with the
% file validation for the camera-ready version.
%
% If you comment hyperref and then uncomment it, you should delete
% ReviewTempalte.aux before re-running LaTeX.
% (Or just hit 'q' on the first LaTeX run, let it finish, and you
%  should be clear).
\usepackage[pagebackref,breaklinks,colorlinks]{hyperref}


% Support for easy cross-referencing
\usepackage[capitalize]{cleveref}
\crefname{section}{Sec.}{Secs.}
\Crefname{section}{Section}{Sections}
\Crefname{table}{Table}{Tables}
\crefname{table}{Tab.}{Tabs.}


%%%%%%%%% PAPER ID  - PLEASE UPDATE
\def\cvprPaperID{*****} % *** Enter the CVPR Paper ID here
\def\confName{ECE 251C}
\def\confYear{Fall 2021}


\begin{document}

%%%%%%%%% TITLE - PLEASE UPDATE
\title{Feature Extraction of ECG signals using Wavelet transform \confName~Final Project report}

\author{Hari Prasad Sankar\\
UCSD\\
Institution1 address\\
{\tt\small hsankar@ucsd.edu}
% For a paper whose authors are all at the same institution,
% omit the following lines up until the closing ``}''.
% Additional authors and addresses can be added with ``\and'',
% just like the second author.
% To save space, use either the email address or home page, not both
\and
Abhishek Khare\\
UCSD\\
First line of institution2 address\\
{\tt\small akhare@ucsd.edu}
}
\maketitle

%%%%%%%%% ABSTRACT
\begin{abstract}
   The ABSTRACT is to be in fully justified italicized text, at the top of the left-hand column, below the author and affiliation information.
   Use the word ``Abstract'' as the title, in 12-point Times, boldface type, centered relative to the column, initially capitalized.
   The abstract is to be in 10-point, single-spaced type.
   Leave two blank lines after the Abstract, then begin the main text.
   Look at previous CVPR abstracts to get a feel for style and length.
\end{abstract}

%%%%%%%%% BODY TEXT
\section{Introduction}
\label{sec:intro}

In today’s age Rising Cardiovascular diseases pegs need of timely diagnosis. Apart from regular health checkups an easy & accessible heart health diagnosis solution is needed to prevent mishaps like heart failure. One such solution can be - to use the pulse data (similar to ECG data) from wearable health monitoring devices like ‘fitbit’ and analyze that data to detect/diagnose an early warning sign which will enable us to predict the onset of a cardiovascular disease. There are various factors which we would have to incorporate such a system – like the power consumption, performance, precision of measurement, desired accuracy of the predictions, etc.
Now, to build a full ECG abnormality detection and classification mechanism, we would have to choose and implement proper techniques to perform the following actions- data preprocessing, feature extraction, and classification. In this project, our main focus would be on the feature extraction step of this design and the usage of wavelet decomposition and analysis techniques - namely - Discrete Wavelet Transform (DWT) and Wavelet Packet Decomposition (WPD) for the feature extraction of ECG signals.
To see how effective DWT and WPD feature extraction techniques are with respect to different classifiers and what impact do they have on classification accuracy, we have performed the experiment where we vary the scale(5,9,10), mother wavelet type (db4, db6, haar, symlet4), technique( DWT/ WPD) and classifier(SVM, KNN, Fine decision Tree)  to obtain the best working combination of these to maximize the accuracy of the system.

%-------------------------------------------------------------------------
\section{Previous Works}
ECG signals have been traditionally been used to identify cardiovascular abnormalities present in the human body. However, pure ECG signals are hard to come by due to the fact that breathing pattern frequencies or electrical interferences corrupt the signal to a great extent. As a result, traditional approaches have relied on several pre-processing to actually make the signal worthy of inputting it further into data processing systems. Two types of abnormalities corrupt the ECG signals, the first one being the baseline wandering and the next one being
Baseline wandering is an artifact present in an ECG signal and the causes for baseline wandering are the respiration of the patient and movement while recording the ECG. These make the signal hard to interpret as there might be additional frequencies other than the desired heart rate frequency alone. The traditional method of removing the baseline wandering is passing the signal into a high pass filter. 111 paper has a section that talks about many such filtering methods for removing baseline wandering.\\
Noise in an ECG signal creeps into the ECG recording if the recording pads are not kept properly, Electromyographic noise, and power line interferences. These noises can be removed by utilizing a low pass notch filter at 50/60Hz usually.  But 111 also suggests further developments that have been used to remove the noise more efficiently.\\
Once the pre-processing is done, different ways are used to extract features which is then fed into the classifier. A classic feature extraction method is the Fourier transform method, peak detection( P-QRS-T wave detection). Except for the Fourier transform technique, other techniques mainly rely on time-based abnormality detections.  In the Peak detection technique, the distance between the peaks acts as the main classification basis which is computed as the distance between two R points in the P-QRS-T wave 222. A simple method of classification is the Principle component analysis of the signal itself though which the classifier will handle by itself. For frequency-based features, performing the FFT followed by the classification based on FFT coefficients is traditionally simple and gives the frequency-based feature classification. 


\section{Proposed Approach}

In our proposed approach, we use DWT and WPD to perform the feature extraction step instead of using conventional feature extraction methods for ECG signals. By using this approach, we aim to optimize the accuracy of the classifier and also provide better (workable) prediction results for low-computational classifiers. Apart from this, feature extraction using this method also eliminates the need for noise removal in the data-preprocessing step, thus, we can say that using this technique still keeps the classification process immune to non-linearities of the signal even after removing the noise filtering process in the data-preprocessing step.
The mentioned Non-linearities of ECG signals like powerline interference will not impact the classification result while using the wavelet transform for feature extraction because when we select the features obtained from wavelet decomposition, we would ignore the data present in trivial frequency bands (containing non-essential data/ noise) and we would not feed that un-essential data to the classifier, thus denoising can be done at the feature extraction step itself. 
Once we are done with the pre-processing, we need to take the DWT/WPD coefficients. Compute the classification (Range of signals/energy) as a parameter and give it as an input to the classifiers which were used in the traditional classification methods in order to compare the output results.
We will perform a comparative study for determining which type of mother wavelet is best suited for this application and up to what scale (levels) of cascading DWT/WPD is useful for this process. furthermore, we would also check if the features extracted using wavelet packet decomposition (WPD) give better results than DWT.
Comparison experiment variables - Mother Wavelet type (Haar, DB, Symlet), Scale of DWT/WPD, Type of Decomposition (WPD or DWT).


\section{Simulation Results, Comparison & Discussion}
Papers, excluding the references section, 
{\bf There will be no extra page charges for \confName\ \confYear.}

Overlength papers will simply not be reviewed.


%-------------------------------------------------------------------------
\section{Conclusion}
The \LaTeX\ style defines a printed ruler which should be present in the version submitted for review.
The ruler is provided in order that reviewers may comment on particular lines in the paper without circumlocution.
If you are preparing a document using a non-\LaTeX\ document preparation system, please arrange for an equivalent ruler to appear on the final output pages.
The presence or absence of the ruler should not change the appearance of any other content on the page.
The camera-ready copy should not contain a ruler.
(\LaTeX\ users may use options of cvpr.sty to switch between different versions.)

\section{Final copy}

You must include your signed IEEE copyright release form when you submit your finished paper.
We MUST have this form before your paper can be published in the proceedings.

Please direct any questions to the production editor in charge of these proceedings at the IEEE Computer Society Press:
\url{https://www.computer.org/about/contact}.


%%%%%%%%% REFERENCES
{\small
\bibliographystyle{ieee_fullname}
\bibliography{egbib}
}

\end{document}
